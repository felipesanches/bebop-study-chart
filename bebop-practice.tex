\documentclass{article}
\usepackage{enumitem,amssymb}
\title{Bebop study chart}
\newlist{todolist}{itemize}{2}
\setlist[todolist]{label=$\square$}
\begin{document}

\maketitle
\begin{itemize}
  \item Things that we practice on a dominant scale (such as Bb7):

  \begin{todolist}
    \item Scale up and down.
    \item Scale in thirds.
    \item Scale in triads.
    \item Scale in chords.
    \item We do all of these scales up and down.
    \item And then all of them with half-step below.
    \item Up and down! All those things.
    \item Then we do pivots, from every degree.
    \item Barry Harris' descending half-step rules...
    \item Starting on root, 3rd, 5th or seven:
    \begin{todolist}
      \item A single half-step (between tonic and flat 7th).
      \item Or 3 half-steps (between: tonic and flat 7th / 3rd and 2nd / 2nd and tonic)
    \end{todolist}
    \item Starting on 2nd, 4th or 6th:
    \begin{todolist}
      \item No half-step.
      \item Or 2 half-steps. (between: 2nd and tonic / tonic and flat 7th)
    \end{todolist}
    \item Then all the rules withing those:
    \begin{todolist}
    \item Start on a note and run up to another note scale-wise, then come back down. Use the rule for the note that we started on.
    \item Starting on a note and going up a 3rd, use the rule of the next note we hit descending.
    \item Go up a triad, use the rule for the top note of the triad.
    \end{todolist}
    \item Barry's cromatic scale. Add cromatic steps between all scale tones, and jumps to next scale not in the single half-tone intervals of the scale. 
    \item Descending rules for triplets:
    \begin{todolist}
    \item 8th note triplet - follow the rule of the note we land on after the triplet.
    \item 16th note triplet - we use the rule for the note we started on.
    \end{todolist}
  \end{todolist}

\end{itemize}
\end{document}
